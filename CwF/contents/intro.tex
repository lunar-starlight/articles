\section{Introduction}

When trying to find models where some chosen non-constructive principles hold or
fail we find that translating the statements using Kripke-Joyal semantics to
produce very long formulae.

For example, consider the schema of instance reduction \cite{Bau22}
\[ \for{x:A}{\exist{y:B}{ψ(y) ⇒ φ(x)}}\text. \]
This translates to, in the Kripke-Joyal semantics, into
\[ \for{I ∈ 𝒞, x ∈ A(I)}{\exist{C\text{ covering}}
    {\for{(j : J → I) ∈ C}{\exist{y ∈ B(J)}{J ⊩ ψ(y) ⇒ φ(x\res j)}}}}\text. \label{eq:inst-red-kj}\]

Furthermore, this is not any simpler for even the simplest of Grothendieck
toposes \(\sh(X)\).
However, there is an alternative to Kripke-Joyal semantics in topological (and
in general localic) toposes. We can simply evaluate the expressions. We define
the interpretation \(\i φ\) as \(⋁_{U ∈ ℋ}U ⊩ φ\) aka the largest open, where
\(φ\) holds. If we then rephrase the Kripke-Joyal semantics in terms of
interpretation, we get the following table
TODO: TABLE

Then instance reduction translates to
\[
  ⋀_{\substack{U ∈ ℋ\\x ∈ A(U)}}⋁_{\substack{V ⊆ U\\y ∈ B(V)}}
  \i{ψ(y)} ⇒ \i{φ(x)}\text.\label{eq:inst-red-int}
\]

This looks much closer to the original, but now we need to reason in the order
structure of \(ℋ\) instead of the meta-logic, like we normally do.

TODO: relate to something idk

There is a lot of bookkeeping of stages in~\ref{eq:inst-red-int}. There exists a
presentation of sheaves that avoids the stratification, where we just have a set
of elements equipped with a map into \(ℋ\), encoding the strata. However, we can
do even better. After all, sheaves are closed under restrictions and gluing, so
wouldn't it be better, if we could ignore those and just work with the
"generators", like we do in group theory?

It turns out the answer to the above is hidden in the work of Michael Fourman
and Dana Scott~\cite{FS79} in the form of what they call \(Ω\)-sets, and what we will
call \emph{Heyting-valued sets}. These encode the representability structure of
sheaves, by giving a set of generators together with a Heyting-valued relation
on pairs of generators, measuring their "degree of equality".
This framework gives us much nicer objects to work with, especially for
colimits. For example, the sum of two sheaves is somewhat complicated. We first
take the simple sum of presheaves, but then we need to add all of the glued
elements. This makes verifying formulae like~\ref{eq:inst-red-kj} for these
elements unnecessarily cumbersome. After all, the sum is \emph{generated} by
elements of it's components. In Heyting-valued sets, the sum is constructed
simply by taking the sum of the underlying sets, with equality defined in the
obvious way. This means that all elements of \(A + B\) are of the form either
\(\inl a\) or \(\inr b\), which we can then reduce to computation purely in
either \(A\) or \(B\).


%%% Local Variables:
%%% mode: latex
%%% TeX-master: "../main"
%%% End:
