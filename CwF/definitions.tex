\directlua{pdf.setminorversion(7)}
\RequirePackage{amsfonts,amsmath,amssymb,amsthm}
\RequirePackage{lmodern}
\RequirePackage{makeidx}
\def\normalsize{}
\RequirePackage{fontspec}
\RequirePackage{stmaryrd}
% \RequirePackage{lualatex-math}
\RequirePackage{mathtools}
\RequirePackage{unicode-math}
\RequirePackage{polyglossia}
\RequirePackage{enumerate}
\RequirePackage{tikz}
\usetikzlibrary{babel}
\usetikzlibrary{cd,positioning}
\usetikzlibrary{angles,quotes}

\usepackage{amsthm}

\DeclareMathOperator{\sh}{Sh}
\undef\fam
\DeclareMathOperator{\fam}{Fam}
\newcommand{\ressymb}{\mathchoice
  {\mkern-5mu↾\mkern-8mu}
  {\mkern-5mu↾\mkern-8mu}
  {↾\mkern-4mu}
  {↾}}
\newcommand{\res}[1]{\ressymb_{#1}}
\newcommand{\cat}[1]{\mathscr{#1}}
\newcommand{\smallcat}[1]{\mathbb{#1}}
\newcommand{\namedcat}[1]{\mathrm{#1}}
\newcommand{\sets}{\namedcat{Sets}}
\newcommand{\hvsets}{\namedcat{HVSets}}
\newcommand{\quot}[2]{{#1}/_{\!#2}}
%\newcommand{\eq}[3][]{#2 =_{#1} #3}
%\newcommand{\eq}[3][]{\left⟦ #2 =_{#1} #3 \right⟧}
\newcommand{\eq}[3][]{#2 \sim_{#1} #3}
\newcommand{\e}[1]{\mathrm{E}{#1}}
\renewcommand{\i}[1]{\left⟦#1\right⟧}
\DeclarePairedDelimiterX{\parens}[1]{(}{)}{#1}
\DeclarePairedDelimiterX{\absolute}[1]{\lvert}{\rvert}{#1}
\def\p{\parens*}
\def\abs{\absolute*}
\def\carr{\absolute*}
\newcommand{\set}[2]{\left\{ #1 \mid #2 \right\}}
\newcommand{\apart}{\ensuremath{\mathrel{\#}}}
\newcommand{\mb}[1]{\mathbf{#1}}
\newcommand{\bb}[1]{\mathbb{#1}}
\newcommand{\mf}[1]{\mathfrak{#1}}
\newcommand{\mc}[1]{\mathcal{#1}}
\DeclareMathOperator{\inl}{inl}
\DeclareMathOperator{\inr}{inr}

\newcommand{\defquantifier}[2]{%
  \expandafter\undef\csname #1\endcsname%
  \expandafter\newcommand\csname #1\endcsname[2]{{#2 ##1.}\;##2}%
}
\defquantifier{for}{\forall}
\defquantifier{exist}{\exists}
\defquantifier{unique}{\exists!}
%\defquantifier{globalen}{\exists ᵍ}
\defquantifier{exact}{ι}
\defquantifier{eventually}{\nabla}
\AtBeginDocument{
\defquantifier{sigma}{\Sigma}
\defquantifier{pi}{\Pi}
\def\tilde{\widetilde}
}
\newcommand{\newdef}[1]{
  \theoremstyle{definition}
  \newtheorem{#1}[theorem]{\MakeUppercase #1}
}
\newcommand{\newex}[2]{
  \theoremstyle{definition}
  \newtheorem{#1*}[theorem]{\MakeUppercase #1}
  \newenvironment{#1}[1][]{
    \begin{#1*}[##1]\renewcommand*{\qedsymbol}{\(#2\)}\pushQED{\qed}
  }{
    \popQED\end{#1*}
  }
}

\newcommand{\newthm}[1]{
  \theoremstyle{plain}
  \newtheorem{#1}[theorem]{\MakeUppercase #1}
}

\newcommand{\newpf}[2]{
  \newenvironment{#1}[1][\MakeUppercase #1]{
    \begin{proof}[##1]\renewcommand*{\qedsymbol}{\(#2\)}
    }{
    \end{proof}
  }
}

\theoremstyle{plain}
\newtheorem{theorem}{Theorem}[section]
\newdef{definition}
\theoremstyle{definition}
\newtheorem*{internal}{Internal Motivation}

\newtheorem{lemma}[theorem]{Lemma}
\newtheorem{proposition}[theorem]{Proposition}
\newtheorem{corollary}[theorem]{Corollary}
\newtheorem{note}[theorem]{Note}

\theoremstyle{definition}
\newtheorem{example}[theorem]{Example}
\newtheorem{fact}[theorem]{Fact}
\newtheorem{intuition}[theorem]{Intuition}
\newtheorem{construction}[theorem]{Construction}

\newcommand{\Psh}[1]{\widehat{#1}}
\newcommand{\Yon}{\mathbf{y}}

\newcommand{\Assoc}{\mathbf{a}}
\newcommand{\Incl}{\mathbf{i}}

\newcommand{\Opens}{\mathcal{O}}


\usepackage{tabularx}
\newcolumntype{C}{>{\centering\arraybackslash}X}
\newcommand{\mordef}[5]{
  \[
  \begin{tabular}{rrcl}
    \(#1 :\) & \(#2\)                     & \(→\) & \(#3\) \\
             & \multicolumn{1}{c}{\(#4\)} & \(↦\) & \multicolumn{1}{c}{\(#5\)}
  \end{tabular}
  \]
}
%%% Local Variables:
%%% mode: latex
%%% TeX-master: "main"
%%% End:
