\section{Heyting-valued Sets}

\def\h{\mathcal{H}}

\subsection{Preliminaries}

\begin{definition}
  A \emph{Heyting-valued set \(A\)} is a carrier set \(\abs A\) equipped with a
  ``Heyting-valued relation'' \(\eq[A]{-}{-} : A×A → \h\) satisfying
  \begin{gather}
    \label{hvs:symm}  \eq x y ≤ \eq y x\\
    \label{hvs:trans} \eq x y ∧ \eq y z ≤ \eq x z
  \end{gather}

  Define \(\e x ≔ \eq x x\).
\end{definition}

\begin{definition}
  A \emph{Heyting-valued morphism \(f : A ↬ B\)} is a ``Heyting-valued,b
  functional relation'', meaning it is a map \(\abs f : \abs A × \abs B → \h\)
  satisfying
  \begin{gather}
    \label{hvm:str}    f(x,y) ≤ \e x ∧ \e y\\
    \label{hvm:well}   \eq {x'} x ∧ f(x, y) ∧ \eq y {y'} ≤ f(x', y')\\
    \label{hvm:unique} f(x, y) ∧ f(x, y') ≤ \eq y {y'}\\
    \label{hvm:total}  \e x ≤ ⋁_{y ∈ B}f(x, y)
  \end{gather}
\end{definition}

\begin{fact}
  The following hold.
  
  \begin{enumerate}
  \item \(f(x,y)∧\eq x {x'} = f(x',y)∧\eq x {x'}\)
  \item \(\e x = ⋁_{y∈A}\eq x y\)
  \end{enumerate}
\end{fact}


\begin{construction}
  The product \(A×B\) is given by \(\abs A × \abs B\) with equality
  \[ \eq{\p{x,y}}{\p{x',y'}} ≔ \eq x {x'} ∧ \eq y {y'}\text. \]

  The projection \(π : A×B ↬ A\) is defined by
  \[ \p{\p{x,y}, x'} ↦ \eq x {x'} ∧ \e y\text. \]
  The other projection is defined similarly
\end{construction}
\begin{proof}
  Symmetry and transitivity of the equality follow clearly.

  Likewise, the projection is clearly strict and well-defined.
  Uniqueness holds via
  \begin{align*}
    π(\p{x,y},x') ∧ π(\p{x,y},x'')
    &= \eq x {x'} ∧ \e y ∧ \eq x {x''} ∧ \e y\\
    &= \eq x {x'} ∧ \eq x {x''} ∧ \e y\\
    &≤ \eq {x'} {x''} ∧ \e y\\
    &≤ \eq {x'} {x''}\text.
  \end{align*}
  Totality holds via
  \begin{align*}
    \e{\p{x,y}} = \e x ∧ \e y = ⋁_{x'∈A}\eq x {x'} ∧ \e y\text.
  \end{align*}
\end{proof}

In the following, I will not write out these proofs. A standard reference is
TODO BorceuxV3.

\begin{construction}
  The \emph{discrete Heyting-valued set \(ΔA\) on \(A\)} has \(A\) as it's
  carrier and equality defined by \(\p{x,y} ↦ ⋁\set{⊤}{x=y}\).
\end{construction}

\begin{fact}
  The empty type \(∅\) is given by the empty carrier.

  The unit type \(\mb 1\) is given by \(\{\ast\}\) with \(\e\ast = ⊤\). This is
  the discrete Heyting-valued set on \(\{\ast\}\).

  The natural numbers object is the discrete Heyting-valued set on \(ℕ\).

  The rational numbers object is the discrete Heyting-valued set on \(ℚ\).
\end{fact}
\begin{note}
  The real numbers object is not necessarily discrete.
\end{note}

\begin{construction}
  The sum \(A+B\) is given by \(\abs A+\abs B\) with equality defined by
  \begin{align*}
    &\eq {\inl x} {\inl x'} ≔ \eq[A] x {x'}\\
    &\eq {\inl x} {\inr y'} ≔ ⊥\\
    &\eq {\inr y} {\inr x'} ≔ ⊥\\
    &\eq {\inr y} {\inr y'} ≔ \eq[B] y {y'}
  \end{align*}

  The inclusion \(ι : A ↬ A+B\) is given by \(\p{x,t} ↦ \eq {\inl x} t\).
  The other inclusion is defined similarly.
\end{construction}

\begin{construction}
  The subobject classifier \(Ω\) is given by \(\h\) with \(∧\) as equality.
\end{construction}

\begin{construction}
  The exponent \(B^A\) is given by
  \[ \set{f : \abs A × \abs B → \h}{\ref{hvm:str}-\ref{hvm:unique}} \]
  with equality
  \[ \p{f,g} ↦ ⋀_{x∈A}⋁_{y∈B}f(x,y) ∧ g(x,y)\text. \]
\end{construction}

\begin{fact}
  The powerobject \(𝒫A = Ω^A\) can be given as the discrete Heyting-valued set
  on \(\set{P : \abs A → Ω}{\eq {x'} x ∧ P(x) ≤ P(x')}\).
\end{fact}

\begin{construction}
  The restriction \(A\res p\) is given by \(\set{x ∈ A}{\e x \between p}\)
\end{construction}


\subsection{CwF structure}

To define what is a family we first define a ``universe''.

\begin{construction}
  Up to size issues the class of all Heyting-valued sets \(\hvsets\) with
  equality \(\p{A,B} ↦ ⋁\set{p∈\h}{A\res p = B\res p}\) is a Heyting-valued set.
\end{construction}

\begin{construction}
  Given a  family \(\p{Bₐ}_{a∈A}\) of Heyting-valued sets, such that
  \(Bₐ\res{\eq a {a'}} = B_{a'}\res{\eq a {a'}}\), the gluing \(⋁B\) has carrier
  \[ \set{b ∈ Bₐ\res{\e a}}{a ∈ A} = ⋃_{a ∈ A}Bₐ\res{\e a} \] and equality
  %\[ \p{b, b'} ↦ ⋁_{a,a' ∈ A} b ∈ Bₐ\res {\eq a {a'}} ∧ b' ∈ B_{a'}\res{\eq a {a'}}
  %  ∧ \eq a {a'} ∧ \eq[Bₐ\res {\eq a {a'}}] b {b'}\text. \]
  \[ \p{b, b'} ↦ ⋁_{a ∈ A} b,b' ∈ Bₐ\res {\e a} ∧ \eq[Bₐ\res {\e a}] b {b'}\text. \]
\end{construction}
\begin{proof}
  Let \(b₀, b₁, b₂ ∈ ⋁B\). Then by construction \(bᵢ ∈ B_{aᵢ}\res{\e {aᵢ}}\) for
  some \(aᵢ ∈ A\).
  \begin{align*}
    \eq {b₀} {b₁}
    = ⋁_{a ∈ A} b₀,b₁ ∈ Bₐ\res {\e a} ∧ \e a ∧ \eq[Bₐ\res {\e a}] {b₀} {b₁}\\
    ≤ ⋁_{a ∈ A} b₁,b₀ ∈ Bₐ\res {\e a} ∧ \e a ∧ \eq[Bₐ\res {\e a}] {b₁} {b₀}\\
    &= \eq {b₁} {b₀}
  \end{align*}
  \begin{align*}
    \eq {b₀} {b₁} ∧ \eq {b₁} {b₂}
    = ⋁_{a,a' ∈ A} b₀,b₁ ∈ Bₐ\res {\e a} ∧ \e a ∧ \eq[Bₐ\res {\e a}] {b₀} {b₁}
    ∧ b₁,b₂ ∈ B_{a'}\res {\e {a'}} ∧ \e {a'} ∧ \eq[B_{a'}\res {\e {a'}}] {b₁} {b₂}\\
    &= ⋁_{a ∈ A} b₀,b₁,b₂ ∈ Bₐ\res {\e a} ∧ \e a
    ∧ \eq[Bₐ\res {\e a}] {b₀} {b₁} ∧ \eq[Bₐ\res {\e a}] {b₁} {b₂}\\
    &≤ ⋁_{a ∈ A} b₀,b₂ ∈ Bₐ\res {\e a} ∧ \e a
    ∧ \eq[Bₐ\res {\e a}] {b₀} {b₂}\\
    &= \eq {b₀} {b₂}
  \end{align*}
\end{proof}

\begin{theorem}
  Morphisms \(f : A ↬ 𝒰\) correspond precisely to maps \(\hat f : A → 𝒰\)
  satisfying \(\hat f(x)\res{\eq x {x'}} = \hat f(x')\res{\eq x {x'}}\).
\end{theorem}
\begin{proof}
  Take \(a ∈ A\). The carrier of \(\hat f(a)\) is the set
  \(\set{x ∈ X}{X ∈ 𝒰 ∧ \e x \between f(a, X)}\) with equality
  \(\p{x,y} ↦ ⋁_{X∋x,\\Y∋y}f(a, X) ∧ f(a, Y) ∧ \eq[X\res{\eq X Y}] x y\).

  Take another \(a' ∈ A\). The restriction \(\hat f(a)\res{\eq a {a'}}\) has as
  carrier
  \[ \set{x ∈ X}{X ∈ 𝒰 ∧ \e x ∧ f(a,x) \between \eq a {a'}}\text. \]
  The condition \(\e x ∧ f(a, X) \between \eq a {a'}\) is equivalent to
  \[ ⊥ < \e x ∧ f(a, X) ∧ \eq a {a'} \text, \]
  so it is also equivalent to \(\e x ∧ f(a', X) \between \eq a {a'}\).

  Thus, \(\hat f(a)\res{\eq a {a'}} = \hat f(a')\res{\eq a {a'}}\).

  In the other direction taking a map \(\hat f : A → 𝒰\) we define
  \(f(a, X) ≔ \e a ∧ \eq X {\hat f(a)}\).
\end{proof}
















\end{document}

%%% Local Variables:
%%% mode: latex
%%% TeX-master: "../main"
%%% End:
